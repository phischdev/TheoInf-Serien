% !TeX spellcheck = en_US
%BEGIN_FOLD
%%%%%%%%%%%%%%%%%%%%%%%%%%%%%%%%%%%%%%%%
% Programming/Coding Assignment
% LaTeX Template
%
% Original author:
% Ted Pavlic (http://www.tedpavlic.com)
%
%
% This template uses a Perl script as an example snippet of code, most other
% languages are also usable. Configure them in the "CODE INCLUSION 
% CONFIGURATION" section.
%
%%%%%%%%%%%%%%%%%%%%%%%%%%%%%%%%%%%%%%%%%
%END_FOLD
%----------------------------------------------------------------------------------------
%BEGIN_FOLD	%PACKAGES AND OTHER DOCUMENT CONFIGURATIONS
%----------------------------------------------------------------------------------------

\documentclass{article}
\usepackage[utf8]{inputenc}

\usepackage[ngerman]{babel}
\usepackage{amsmath}
\usepackage{amssymb}
\usepackage{mathtools}
\usepackage{amsfonts}
\usepackage{fancyhdr} % Required for custom headers
\usepackage{lastpage} % Required to determine the last page for the footer
\usepackage{extramarks} % Required for headers and footers
\usepackage[usenames,dvipsnames]{color} % Required for custom colors
\usepackage{graphicx} % Required to insert images
\usepackage{listings} % Required for insertion of code
\usepackage{courier} % Required for the courier font
\usepackage{enumerate} % used for enumerate args
\usepackage{multicol} % columns
\usepackage{calrsfs} % calligraphic font

\usepackage{pgf} 
\usepackage{tikz}
\usepackage{forest} % treees :D
\usetikzlibrary{arrows,automata} %for FSM
\usepackage{mathtools}

% Custom commands
\DeclareMathOperator{\Kl}{Kl} %Klassen von Zuständen
\newcommand{\Ebool}{\{0, 1\}}
\DeclarePairedDelimiter{\ceil}{\lceil}{\rceil}
\DeclareMathAlphabet{\pazocal}{OMS}{zplm}{m}{n}
% Shamelessly copied from http://tex.stackexchange.com/questions/43008/absolute-value-symbols
\DeclarePairedDelimiter\abs{\lvert}{\rvert} % nice |x|
\DeclarePairedDelimiter\norm{\lVert}{\rVert} % nice ||x||
% Swap the definition of \abs* and \norm*, so that \abs
% and \norm resizes the size of the brackets, and the 
% starred version does not.
%END_FOLD
%----------------------------------------------------------------------------------------
%BEGIN_FOLD
\makeatletter
\let\oldabs\abs
\def\abs{\@ifstar{\oldabs}{\oldabs*}}
\let\oldnorm\norm
\def\norm{\@ifstar{\oldnorm}{\oldnorm*}}
\makeatother


% Margins
\topmargin=-0.45in
\evensidemargin=0in
\oddsidemargin=0in
\textwidth=6.5in
\textheight=9.0in
\headsep=0.25in

\linespread{1.1} % Line spacing

% Set up the header and footer
\pagestyle{fancy}
\lhead{\hmwkAuthorName} % Top left header
%\chead{\hmwkClass\ (\hmwkClassInstructor\): \hmwkTitle} % Top center head
%\rhead{\firstxmark} % Top right header
\rhead{}
\lfoot{\lastxmark} % Bottom left footer
\cfoot{} % Bottom center footer
\rfoot{Seite\ \thepage\ von\ \protect\pageref{LastPage}} % Bottom right footer
\renewcommand\headrulewidth{0.4pt} % Size of the header rule
\renewcommand\footrulewidth{0.4pt} % Size of the footer rule

\setlength\parindent{0pt} % Removes all indentation from paragraphs

%----------------------------------------------------------------------------------------
%	CODE INCLUSION CONFIGURATION
%----------------------------------------------------------------------------------------

\definecolor{MyDarkGreen}{rgb}{0.0,0.4,0.0} % This is the color used for comments
\lstloadlanguages{Pascal} % Load Pascal syntax for listings, for a list of other languages supported see: ftp://ftp.tex.ac.uk/tex-archive/macros/latex/contrib/listings/listings.pdf
\lstset{language=Perl, % Use Pascal in this example
        frame=single, % Single frame around code
        basicstyle=\small\ttfamily, % Use small true type font
        keywordstyle=[1]\color{Blue}\bf, % Pascal functions bold and blue
        keywordstyle=[2]\color{Purple}, % Pascal function arguments purple
        keywordstyle=[3]\color{Blue}\underbar, % Custom functions underlined and blue
        identifierstyle=, % Nothing special about identifiers                                         
        commentstyle=\usefont{T1}{pcr}{m}{sl}\color{MyDarkGreen}\small, % Comments small dark green courier font
        stringstyle=\color{Purple}, % Strings are purple
        showstringspaces=false, % Don't put marks in string spaces
        tabsize=5, % 5 spaces per tab
        %
        % Put standard Pascal functions not included in the default language here
        morekeywords={rand},
        %
        % Put Pascal function parameters here
        morekeywords=[2]{on, off, interp},
        %
        % Put user defined functions here
        morekeywords=[3]{test},
        %
        morecomment=[l][\color{Blue}]{...}, % Line continuation (...) like blue comment
        numbers=left, % Line numbers on left
        firstnumber=1, % Line numbers start with line 1
        numberstyle=\tiny\color{Blue}, % Line numbers are blue and small
        stepnumber=5 % Line numbers go in steps of 5
}

% Creates a new command to include a perl script, the first parameter is the filename of the script (without .p), the second parameter is the caption
\newcommand{\pascalscript}[2]{
\begin{itemize}
\item[]\lstinputlisting[caption=#2,label=#1]{#1.p}
\end{itemize}
}

%----------------------------------------------------------------------------------------
%	DOCUMENT STRUCTURE COMMANDS
%	Skip this unless you know what you're doing
%----------------------------------------------------------------------------------------

% Header and footer for when a page split occurs within a problem environment
%\newcommand{\enterProblemHeader}[1]{
%\nobreak\extramarks{#1}{#1 continued on next page\ldots}\nobreak
%\nobreak\extramarks{#1 (continued)}{#1 continued on next page\ldots}\nobreak
%}

% Header and footer for when a page split occurs between problem environments
%\newcommand{\exitProblemHeader}[1]{
%\nobreak\extramarks{#1 (continued)}{#1 continued on next page\ldots}\nobreak
%\nobreak\extramarks{#1}{}\nobreak
%}

\setcounter{secnumdepth}{0} % Removes default section numbers
\newcounter{homeworkProblemCounter} % Creates a counter to keep track of the number of problems

\newcommand{\homeworkProblemName}{}
%END_FOLD
\newenvironment{homeworkProblem}[1][Aufgabe \arabic{homeworkProblemCounter}]
%BEGIN_FOLD
{ % Makes a new environment called homeworkProblem which takes 1 argument (custom name) but the default is "Problem #"

\stepcounter{homeworkProblemCounter} % Increase counter for number of problems
\renewcommand{\homeworkProblemName}{#1} % Assign \homeworkProblemName the name of the problem
\section{\homeworkProblemName} % Make a section in the document with the custom problem count
%\enterProblemHeader{\homeworkProblemName} % Header and footer within the environment
}{
%\exitProblemHeader{\homeworkProblemName} % Header and footer after the environment
}

\newcommand{\problemAnswer}[1]{ % Defines the problem answer command with the content as the only argument
\noindent\framebox[\columnwidth][c]{\begin{minipage}{0.98\columnwidth}#1\end{minipage}} % Makes the box around the problem answer and puts the content inside
}

\newcommand{\homeworkSectionName}{}
\newenvironment{homeworkSection}[1]{ % New environment for sections within homework problems, takes 1 argument - the name of the section
\renewcommand{\homeworkSectionName}{#1} % Assign \homeworkSectionName to the name of the section from the environment argument
\subsection{\homeworkSectionName} % Make a subsection with the custom name of the subsection
%\enterProblemHeader{\homeworkProblemName\ [\homeworkSectionName]} % Header and footer within the environment
}{
%\enterProblemHeader{\homeworkProblemName} % Header and footer after the environment
}
%END_FOLD
%----------------------------------------------------------------------------------------
%BEGIN_FOLD %NAME AND CLASS SECTION
%----------------------------------------------------------------------------------------

\newcommand{\hmwkTitle}{Blatt} % Assignment title
\newcommand{\hmwkDueDate}{9.\ Oktober\ 2015} % Due date
\newcommand{\hmwkClass}{Theoretische Informatik} % Course/class
\newcommand{\hmwkClassInstructor}{} % Teacher/lecturer
\newcommand{\hmwkAuthorName}{Linus Fessler, Markus Hauptner, Philipp Schimmelfennig} % Your name
\newcommand{\hmwkNumber}{6}
\newcommand{\hmwkAssiInfo}{Assistent: Sacha Krug, CHN D 42}
%END_FOLD
%----------------------------------------------------------------------------------------
%BEGIN_FOLD
%----------------------------------------------------------------------------------------
%	TITLE PAGE
%----------------------------------------------------------------------------------------

\title{
\vspace{2in}
\textmd{\textbf{\hmwkClass:\ \hmwkTitle\ \hmwkNumber}}\\
\normalsize\vspace{0.1in}\small{Abgabe\ bis\ \hmwkDueDate}
\\\hmwkAssiInfo
\\
\vspace{0.1in}\large{\textit{\hmwkClassInstructor}
\vspace{3in}
}}
\author{\textbf{\hmwkAuthorName}}
\date{} % Insert date here if you want it to appear below your name

%----------------------------------------------------------------------------------------
%END_FOLD
\begin{document}
%BEGIN_FOLD
\maketitle

%----------------------------------------------------------------------------------------
%	TABLE OF CONTENTS
%----------------------------------------------------------------------------------------

%\setcounter{tocdepth}{1} % Uncomment this line if you don't want subsections listed in the ToC
%END_FOLD
%----------------------------------------------------------------------------------------
\addtocounter{homeworkProblemCounter}{15}
%----------------------------------------------------------------------------------------
\newpage
%\tableofcontents
%\newpage

%----------------------------------------------------------------------------------------
%	Aufgabe S1
%----------------------------------------------------------------------------------------

\begin{homeworkProblem}
	Wir wollen zeigen, dass $L_{q_i} \not\in \pazocal{L}_{R}$, also nicht rekursiv ist. Dazu machen wir einen Widerspruchsbeweis. \\
	Annahme: $L_{qi}$ sei rekursiv. Wir zeigen $L_u \leq_R L_{q_i}$.
	
	\begin{center}
		\begin{tikzpicture}
		
		\draw (0,0) rectangle (13.4,4);
		\draw (0.5,1) rectangle (2,3);
		\draw (4.3,1) rectangle (5.7,3);
		\draw (8.7,1) rectangle (11,3);
		\draw (0.25,0.75) rectangle (5.9,3.25);
		
		\draw[thick,->] (-1,2) to node [above left] {$w \in \{0,1,\#\}^* $} (0.5,2);
		\draw[thick,->] (2,2.5) to node [above, scale=0.75] {$w=\text{Kod}(M)\#x$} (4.3,2.5);
		\draw[thick,->] (2,1.5) to node [align=left, scale=0.75] {Nummer des\\Zustands $q_{accept}$} (4.3,1.5);
		\draw[thick,->] (5.7,2) to node [above, align=right, scale=0.75] {$w'=\text{Kod}(M)\#x\#0^i$} (8.7,2);
		
		\draw[thick,->] (11,2.5) to node [above, align=right, scale=0.9] {akzeptiert $w'$\quad $w \in L_u$} (15,2.5);
		\draw[thick,->] (11,1.5) to node [above, align=right, scale=0.9] {verwirft $w'$\quad\;\; $w \notin L_u$} (15,1.5);
		
		\draw[thick,-] (1.3,1) -- (1.3,0.5);
		\draw[thick,-] (1.3,0.5) to node [below, align=left]{$w\not=\text{Kod}(M)\#x$} (11.5,0.5);
		\draw[thick,->] (11.5,0.5) -- (11.5,1.5);
		
		\draw (1.25,2) node {$Syntax$};
		\draw (5.0,2) node {$concat$};
		\draw (9.9,2) node [align=center] {$M_{q_i}$\\\\$L(M_{q_i})=L_{q_i}$};
		\draw (6,4.35) node {Algorithums $B$ für $L_U$};
		\draw (3.2,3.4) node {Algorithmus A};
		
		\end{tikzpicture}
	\end{center}
	Für ein Wort $w$ entscheiden wir zuerst, ob die Syntax einem Wort in $L_u$ entspricht. Falls nein, ist $w \not\in L_u$.\\
	Falls ja, wählen wir als $i$ die Nummer des Zustands $q_\text{accept}$ in der Kodierung von $M$ und erzeugen daraus $w'$.
	Falls die Anzahl Zustände der TM $M$ nicht $\geq i+1$ ist, verwerfen wir w (diese Arbeit führt der Algorithmus $A$ aus).
	Ansonsten fahren wir wie folgt fort:
	Da eine TM aus $q_\text{accept}$ nicht mehr herausgeht, ist $w \in L_u$, falls $M_{q_i}$ $w'$ akzeptiert, also $M$ den $i$-ten Zustand erreicht.
	Falls $M_{q_i}$ $w'$ verwirft, akzeptiert $M$ also $w$ nicht.
	
	Da A immer hält und nach Annahme $M_{q_i}$ immer hält (da rekursiv), hält auch B immer. Also gilt $L_u \leq_R L_{q_i}$.\\
	Aus $L_{q_i} \in \pazocal{L}_R$ folgt also $L_u \in \pazocal{L}_R$. Aber wir wissen $L_u \not\in \pazocal{L}_R$. Das ist ein Widerspruch, womit $L_{qi} \not\in \pazocal{L}_R$ gilt.
\end{homeworkProblem}

\begin{homeworkProblem}
	Wir wollen zeigen, dass ${L_{q_i}}'$ nicht in $\pazocal{L}_R$ ist.
	Dazu genügt es nach \textit{Lemma 5.4} zu zeigen, dass ${({L_{q_i}}')}^C \not\in \pazocal{L}_R$.\\
	\[
		{({L_{q_i}}')}^C = 
		\left\{ 
			\begin{array}{lll}
				(1) & w \not= \mbox{Kod}(M)\#x\#0^i \\
				(2) & w     = \mbox{Kod}(M)\#x\#0^i \mbox{ und M hat weniger als $i+1$ Zustände}\\
				(3) & w     = \mbox{Kod}(M)\#x\#0^i \mbox{, M hat mehr als $i$ Zustände, M erreicht den $i$-ten Zustand nicht}\\
			\end{array}
		\right.
	\]
	Um zu zeigen, dass ${({L_{q_i}}')}^C \not\in \pazocal{L}_R$ benutzen wir einen Widerspruchsbeweis.\\
	Annahme: $L_H$ ist rekursiv. Wir zeigen $L_H \leq_R {({L_{q_i}}')}^C$.\\
	\begin{center}
		\begin{tikzpicture}
		
		\draw (0,0.3) rectangle (13.4,4);
		\draw (0.5,1.4) rectangle (2,2.6);
		\draw (4.6,1.5) rectangle (5.3,2.5);
		\draw (7.7, 1.7) rectangle (9.3, 3.7);
		\draw (8.7,1) rectangle (11,3);
		\draw (0.2,0.8) rectangle (5.9,3.2);
		
		\draw[thick,->] (-1,2) to node [above left] {$w \in \{0,1,\#\}^* $} (0.5,2);
		\draw[thick,->] (2,2) to node [above, scale=0.75] {$w=\text{Kod}(M)\#x\#0^i$} (4.6,2);
		\draw[thick,->] (5.3,2.3) to node [above, align=left, scale=0.75] {$w'=\text{Kod}(M)\#x$} (7.7,2.3);
		
		\draw[thick,->] (11,2.5) to node [align=right, scale=0.9] {akzeptiert $w'$\quad $w \in L_u$} (15,2.5);
		\draw[thick,->] (11,1.5) to node [align=right, scale=0.9] {verwirft $w'$\quad $w \notin L_u$} (15,1.5);
		
		\draw[thick,-] (1.3,1) -- (1.3,-0.4);
		\draw[thick,->] (1.3,-0.4) to node [above, align=left, scale=0.9]{$(1)$ oder $(2)$ akzeptieren} (5,-0.4);
		%\draw[thick,->] (11.5,0.5) -- (11.5,1.5);
		
		\draw (1.25,2) node {$Syntax$};
		\draw (8.5,3) node [align=center] {$L_H$};
		\draw (9.9,2) node [align=center] {$M_{q_i}$\\\\$L(M_{q_i})=L_{q_i}$};
		\draw (6,4.35) node {Algorithums $B$ für $L_U$};
		\draw (3.2,3.4) node {Algorithmus A};
		\end{tikzpicture}
	\end{center}
	Für ein Wort $w$ bestimmt Algorithmus $A$ zunächst seine Syntax. Ist diese nicht passend für ein Wort in $L_u$ $(1)$ wird das Wort akzeptiert. Ist diese passend aber hat M weniger als $i+1$ Zustände $(2)$ wird das Wort akzeptiert.\\
	Nach der Annahme ist $L_H$ rekursiv, daher kann der Algorithmus A eine TM simulieren, die prüft, ob $w'=\text{Kod}(M)\#x \in L_H$ und damit ob $M$ auf $x$ hält. Hält $M$ nicht, ist $(3)$ erfüllt und $w$ wird akzeptiert. Hält $M$, so kann $M$ von einer TM $M_M$in endlicher Zeit simuliert werden. Dabei wird festgehalten, ob $M$ den $i$-ten Zustand mindestens einmal erreicht. Falls ja, so wird $w$ verworfen, falls nein, ist $(3)$ erfüllt und $w$ wird akzeptiert.\\
	
	B hält also immer. 
\end{homeworkProblem}
\begin{homeworkProblem}
	Wir wollen zeigen, dass $L_{Eq, \lambda} \not\in \pazocal{L}_R$. Dazu nehmen wir an, dass $L_{Eq, \lambda}$ rekursiv ist, und zeigen für $L_{H, \lambda} = \{\text{Kod}(M) \mid x \in \{0, 1\}^* \text{ und $M$ hält auf }\lambda\}$, dass $L_{H, \lambda} \leq_R L_{Eq, \lambda}$.
	
	\begin{center}
		\begin{tikzpicture}
		
		\draw (0,0) rectangle (13.4,4);
		\draw (0.5,1) rectangle (2,3);
		\draw (4.3,1) rectangle (5.7,3);
		\draw (8.7,1) rectangle (11,3);
		\draw (0.25,0.8) rectangle (5.9,3.2);
		
		\draw[thick,->] (-1,2) to node [above left] {$w \in \{0,1,\#\}^* $} (0.5,2);
		\draw[thick,->] (2,2) to node [above, scale=0.85] {$w=\text{Kod}(M)$} (4.3,2);
		\draw[thick,->] (5.7,2) to node [above, align=right, scale=0.7] {$w'=\text{Kod}(M)\#\text{Kod}(M)$} (8.7,2);
		
		\draw[thick,->] (11,2.5) to node [above, align=right, scale=0.9] {akzeptiert $w'$\quad $w \in L_{H,\lambda}$} (15,2.5);
		\draw[thick,->] (11,1.5) to node [above, align=right, scale=0.9] {verwirft $w'$\quad\;\; $w \notin L_{H,\lambda}$} (15,1.5);
		
		\draw[thick,-] (1.3,1) -- (1.3,0.5);
		\draw[thick,-] (1.3,0.5) to node [below, align=left]{$w\not=\text{Kod}(M)$} (11.5,0.5);
		\draw[thick,->] (11.5,0.5) -- (11.5,1.5);
		
		\draw (1.25,2) node {$Syntax$};
		\draw (5.0,2) node {$concat$};
		\draw (9.9,2) node [align=center] {$TM$ $E$\\\\$L(E)=L_{EQ,\lambda}$};
		\draw (6,4.35) node {Algorithums $B$ für $L_{H,\lambda}$};
		\draw (3.2,3.4) node {Algorithmus $A$};
		
		\end{tikzpicture}
	\end{center}
	
	Für ein Wort $w$ testet der Algorithmus $A$ zunächst die Syntax, ob $w = \text{Kod}(M)$ für eine TM $M$, sonst wird $w$ von B verworfen und $w \notin L_{H,\lambda}$.\\
	Ist hingegen $w = \text{Kod}(M)$, konstruiert der Algorithmus $A$ $w'=\text{Kod}(M)\#\text{Kod}(M)$ als Spezialfall von $\text{Kod}(M)\#\text{Kod}(\overline{M})$ und gibt es als Eingabe für die TM $E$ weiter. Verwirft $E$ $w'$, hat $M$ nicht auf $\lambda$ gehalten. Also ist $w \not\in L_{H, \lambda}$. Akzeptiert $E$, dann gilt die Tautologie $\lambda \in L(M) \leftrightarrow \lambda \in L(M)$. Um zu sagen, dass $\lambda \in L(M)$ oder $\lambda \not\in L(M)$, muss $M$ auf $\lambda$ gehalten haben.
	
	Nach der Annahme hält $E$ immer, und damit auch $B$. Also gilt $L_{H, \lambda} \leq_R L_{Eq, \lambda}$.\\
	Aus $L_{Eq, \lambda} \in \pazocal{L}_R$ folgt also $L_{H, \lambda} \in \pazocal{L}_R$. Aber wir wissen $L_{H, \lambda} \not\in \pazocal{L}_R$. Damit haben wir unseren Widerpsruch und die Annahme war falsch $\Rightarrow L_{Eq, \lambda} \not\in \pazocal{L}_R$.
\end{homeworkProblem}

\end{document}