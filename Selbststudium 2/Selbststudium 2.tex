%BEGIN_FOLD
%%%%%%%%%%%%%%%%%%%%%%%%%%%%%%%%%%%%%%%%
% Programming/Coding Assignment
% LaTeX Template
%
% Original author:
% Ted Pavlic (http://www.tedpavlic.com)
%
%
% This template uses a Perl script as an example snippet of code, most other
% languages are also usable. Configure them in the "CODE INCLUSION 
% CONFIGURATION" section.
%
%%%%%%%%%%%%%%%%%%%%%%%%%%%%%%%%%%%%%%%%%
%END_FOLD
%----------------------------------------------------------------------------------------
%BEGIN_FOLD	%PACKAGES AND OTHER DOCUMENT CONFIGURATIONS
%----------------------------------------------------------------------------------------

\documentclass{article}
\usepackage[utf8]{inputenc}

\usepackage[ngerman]{babel}
\usepackage{amsmath}
\usepackage{amssymb}
\usepackage{mathtools}
\usepackage{amsfonts}
\usepackage{fancyhdr} % Required for custom headers
\usepackage{lastpage} % Required to determine the last page for the footer
\usepackage{extramarks} % Required for headers and footers
\usepackage[usenames,dvipsnames]{color} % Required for custom colors
\usepackage{graphicx} % Required to insert images
\usepackage{listings} % Required for insertion of code
\usepackage{courier} % Required for the courier font
\usepackage{enumerate} % used for enumerate args
\usepackage{multicol} % columns

\usepackage{pgf} 
\usepackage{tikz}
\usepackage{forest} % treees :D
\usetikzlibrary{arrows,automata} %for FSM
\usetikzlibrary{positioning}
\usepackage{mathtools}

% Custom commands
\DeclareMathOperator{\Kl}{Kl} %Klassen von Zuständen
\newcommand{\Ebool}{\{0, 1\}}
\DeclarePairedDelimiter{\set}{\{}{\}}
\DeclarePairedDelimiter{\ceil}{\lceil}{\rceil}
% Shamelessly copied from http://tex.stackexchange.com/questions/43008/absolute-value-symbols
\DeclarePairedDelimiter\abs{\lvert}{\rvert} % nice |x|
\DeclarePairedDelimiter\norm{\lVert}{\rVert} % nice ||x||
% Swap the definition of \abs* and \norm*, so that \abs
% and \norm resizes the size of the brackets, and the 
% starred version does not.
%END_FOLD
%----------------------------------------------------------------------------------------
%BEGIN_FOLD
\makeatletter
\let\oldabs\abs
\def\abs{\@ifstar{\oldabs}{\oldabs*}}
\let\oldnorm\norm
\def\norm{\@ifstar{\oldnorm}{\oldnorm*}}
\makeatother


% Margins
\topmargin=-0.45in
\evensidemargin=0in
\oddsidemargin=0in
\textwidth=6.5in
\textheight=9.0in
\headsep=0.25in

\linespread{1.1} % Line spacing

% Set up the header and footer
\pagestyle{fancy}
\lhead{\hmwkAuthorName} % Top left header
%\chead{\hmwkClass\ (\hmwkClassInstructor\): \hmwkTitle} % Top center head
%\rhead{\firstxmark} % Top right header
\rhead{}
\lfoot{\lastxmark} % Bottom left footer
\cfoot{} % Bottom center footer
\rfoot{Seite\ \thepage\ von\ \protect\pageref{LastPage}} % Bottom right footer
\renewcommand\headrulewidth{0.4pt} % Size of the header rule
\renewcommand\footrulewidth{0.4pt} % Size of the footer rule

\setlength\parindent{0pt} % Removes all indentation from paragraphs

%----------------------------------------------------------------------------------------
%	CODE INCLUSION CONFIGURATION
%----------------------------------------------------------------------------------------

\definecolor{MyDarkGreen}{rgb}{0.0,0.4,0.0} % This is the color used for comments
\lstloadlanguages{Pascal} % Load Pascal syntax for listings, for a list of other languages supported see: ftp://ftp.tex.ac.uk/tex-archive/macros/latex/contrib/listings/listings.pdf
\lstset{language=Perl, % Use Pascal in this example
	frame=single, % Single frame around code
	basicstyle=\small\ttfamily, % Use small true type font
	keywordstyle=[1]\color{Blue}\bf, % Pascal functions bold and blue
	keywordstyle=[2]\color{Purple}, % Pascal function arguments purple
	keywordstyle=[3]\color{Blue}\underbar, % Custom functions underlined and blue
	identifierstyle=, % Nothing special about identifiers                                         
	commentstyle=\usefont{T1}{pcr}{m}{sl}\color{MyDarkGreen}\small, % Comments small dark green courier font
	stringstyle=\color{Purple}, % Strings are purple
	showstringspaces=false, % Don't put marks in string spaces
	tabsize=5, % 5 spaces per tab
	%
	% Put standard Pascal functions not included in the default language here
	morekeywords={rand},
	%
	% Put Pascal function parameters here
	morekeywords=[2]{on, off, interp},
	%
	% Put user defined functions here
	morekeywords=[3]{test},
	%
	morecomment=[l][\color{Blue}]{...}, % Line continuation (...) like blue comment
	numbers=left, % Line numbers on left
	firstnumber=1, % Line numbers start with line 1
	numberstyle=\tiny\color{Blue}, % Line numbers are blue and small
	stepnumber=5 % Line numbers go in steps of 5
}

% Creates a new command to include a perl script, the first parameter is the filename of the script (without .p), the second parameter is the caption
\newcommand{\pascalscript}[2]{
	\begin{itemize}
		\item[]\lstinputlisting[caption=#2,label=#1]{#1.p}
	\end{itemize}
}

%----------------------------------------------------------------------------------------
%	DOCUMENT STRUCTURE COMMANDS
%	Skip this unless you know what you're doing
%----------------------------------------------------------------------------------------

% Header and footer for when a page split occurs within a problem environment
%\newcommand{\enterProblemHeader}[1]{
%\nobreak\extramarks{#1}{#1 continued on next page\ldots}\nobreak
%\nobreak\extramarks{#1 (continued)}{#1 continued on next page\ldots}\nobreak
%}

% Header and footer for when a page split occurs between problem environments
%\newcommand{\exitProblemHeader}[1]{
%\nobreak\extramarks{#1 (continued)}{#1 continued on next page\ldots}\nobreak
%\nobreak\extramarks{#1}{}\nobreak
%}

\setcounter{secnumdepth}{0} % Removes default section numbers
\newcounter{homeworkProblemCounter} % Creates a counter to keep track of the number of problems

\newcommand{\homeworkProblemName}{}
%END_FOLD
\newenvironment{homeworkProblem}[1][Aufgabe S\arabic{homeworkProblemCounter}]
%BEGIN_FOLD
{ % Makes a new environment called homeworkProblem which takes 1 argument (custom name) but the default is "Problem #"
	
	\stepcounter{homeworkProblemCounter} % Increase counter for number of problems
	\renewcommand{\homeworkProblemName}{#1} % Assign \homeworkProblemName the name of the problem
	\section{\homeworkProblemName} % Make a section in the document with the custom problem count
	%\enterProblemHeader{\homeworkProblemName} % Header and footer within the environment
}{
%\exitProblemHeader{\homeworkProblemName} % Header and footer after the environment
}

\newcommand{\problemAnswer}[1]{ % Defines the problem answer command with the content as the only argument
	\noindent\framebox[\columnwidth][c]{\begin{minipage}{0.98\columnwidth}#1\end{minipage}} % Makes the box around the problem answer and puts the content inside
}

\newcommand{\homeworkSectionName}{}
\newenvironment{homeworkSection}[1]{ % New environment for sections within homework problems, takes 1 argument - the name of the section
	\renewcommand{\homeworkSectionName}{#1} % Assign \homeworkSectionName to the name of the section from the environment argument
	\subsection{\homeworkSectionName} % Make a subsection with the custom name of the subsection
	%\enterProblemHeader{\homeworkProblemName\ [\homeworkSectionName]} % Header and footer within the environment
}{
%\enterProblemHeader{\homeworkProblemName} % Header and footer after the environment
}
%END_FOLD
%----------------------------------------------------------------------------------------
%BEGIN_FOLD %NAME AND CLASS SECTION
%----------------------------------------------------------------------------------------

\newcommand{\hmwkTitle}{Selbststudium - Blatt} % Assignment title
\newcommand{\hmwkDueDate}{20.\ November\ 2015} % Due date
\newcommand{\hmwkClass}{Theoretische Informatik} % Course/class
\newcommand{\hmwkClassInstructor}{} % Teacher/lecturer
\newcommand{\hmwkAuthorName}{Linus Fessler, Markus Hauptner, Philipp Schimmelfennig} % Your name
\newcommand{\hmwkNumber}{2}
\newcommand{\hmwkAssiInfo}{Assistent: Sacha Krug, CHN D 42}
%END_FOLD
%----------------------------------------------------------------------------------------
%BEGIN_FOLD
%----------------------------------------------------------------------------------------
%	TITLE PAGE
%----------------------------------------------------------------------------------------

\title{
	\vspace{2in}
	\textmd{\textbf{\hmwkClass:\ \hmwkTitle\ \hmwkNumber}}\\
	\normalsize\vspace{0.1in}\small{Abgabe\ bis\ \hmwkDueDate}
	\\\hmwkAssiInfo
	\\
	\vspace{0.1in}\large{\textit{\hmwkClassInstructor}
		\vspace{3in}
	}}
	\author{\textbf{\hmwkAuthorName}}
	\date{} % Insert date here if you want it to appear below your name
	
	%----------------------------------------------------------------------------------------
	%END_FOLD
\begin{document}
	%BEGIN_FOLD
	\maketitle
	
	%----------------------------------------------------------------------------------------
	%	TABLE OF CONTENTS
	%----------------------------------------------------------------------------------------
	
	%\setcounter{tocdepth}{1} % Uncomment this line if you don't want subsections listed in the ToC
	%END_FOLD
	%----------------------------------------------------------------------------------------
	\addtocounter{homeworkProblemCounter}{3}
	%----------------------------------------------------------------------------------------
	\newpage
	%\tableofcontents
	%\newpage
	
	%------------------Aufgabe S4---------------------------------
	
	\begin{homeworkProblem}
		Für die Konstruktion des äquivalenten EA werden wir folgendermassen vorgehen. Zuerst erstellen wir analog zu 3.2.3 einen $\epsilon$-NEA und wandeln ihn dann nach 2.5.5 in einen EA um, der die gleiche Sprache wie der $\epsilon$-NEA akzeptiert.
		
		Für den regulären Ausdruck $E=\mathbf{((01)^* + 1)0}$ konstruieren wir zuerst einen $\epsilon$-NEA für $\mathbf{01}$:
		
		\begin{center}
		\begin{tikzpicture}[->,>=stealth',shorten >=1pt,auto,node distance=1.25cm,
		semithick,every node/.style={scale=1}]
		\tikzstyle{every state}=[minimum size=10pt,circle,text=black]
		
		\node[state]				(1) 			 		{};
		\node[state]				(2) [right of=1]		{};
		\node[state]				(3) [right of=2]		{};
		
		\path
		(1)		edge [below]				node {$\mathbf{0}$}	(2)
		(2)		edge [below]				node {$\mathbf{1}$}	(3)
		;
		\end{tikzpicture}
		\end{center}
		
		Nun erweitern wir ihn für $\mathbf{(01)^*}$:	
		
		\begin{center}
		\begin{tikzpicture}[->,>=stealth',shorten >=1pt,auto,node distance=1.25cm,
		semithick,every node/.style={scale=1}]
		\tikzstyle{every state}=[minimum size=10pt,circle,text=black]
		
		\node[state]				(0)             		{};
		\node[state]				(1) [right of=0]		{};
		\node[state]				(2) [right of=1]		{};
		\node[state]				(3) [right of=2]		{};
		\node[state]				(4) [right of=3] 		{};
		
		\path
		(0)		edge 						node {$\epsilon$}	(1)
		(1)		edge [below]				node {$\mathbf{0}$}	(2)
		(2)		edge [below]				node {$\mathbf{1}$}	(3)
		(3)		edge [bend right=50,above]	node {$\epsilon$}	(1)
		(0)		edge [bend right=40,below]	node {$\epsilon$}	(4)
		(3)		edge 						node {$\epsilon$}	(4)
		;
		\end{tikzpicture}
		\end{center}
		
		Für $\mathbf{(01)^*+1}$:	
				
		\begin{center}
		\begin{tikzpicture}[->,>=stealth',shorten >=1pt,auto,node distance=1.25cm,
		semithick,every node/.style={scale=1}]
		\tikzstyle{every state}=[minimum size=10pt,circle,text=black]
		
		\node[state]				(5) 					{};
		\node[state]				(0) [above right of=5] 	{};
		\node[state]				(1) [right of=0]		{};
		\node[state]				(2) [right of=1]		{};
		\node[state]				(3) [right of=2]		{};
		\node[state]				(4) [right of=3] 		{};
		\node[state]				(6) [below right of=5]	{};
		\node[state]				(8) [below right of=4] 	{};
		\node[state]				(7) [below left of=8]	{};
		
		\path
		(0)		edge 						node {$\epsilon$}	(1)
		(1)		edge [below]				node {$\mathbf{0}$}	(2)
		(2)		edge [below]				node {$\mathbf{1}$}	(3)
		(3)		edge [bend right=50,above]	node {$\epsilon$}	(1)
		(0)		edge [bend right=40,below]	node {$\epsilon$}	(4)
		(3)		edge 						node {$\epsilon$}	(4)
		(4)		edge 						node {$\epsilon$}	(8)
		(5)		edge 						node {$\epsilon$}	(0)
		(5)		edge [below left]			node {$\epsilon$}	(6)
		(6)		edge [below]				node {$\mathbf{1}$}	(7)
		(7)		edge [below right]			node {$\epsilon$}	(8)
		;
		\end{tikzpicture}
		\end{center}
		
		Und noch für $\mathbf{\big((01)^*+1\big)0}$:	
						
		\begin{center}
		\begin{tikzpicture}[->,>=stealth',shorten >=1pt,auto,node distance=1.75cm,
		semithick,every node/.style={scale=1}]
		\tikzstyle{every state}=[minimum size=20pt,circle,text=black]
		
		\node[initial,state]		(5) 					{$q_0$};
		\node[state]				(0) [above right of=5] 	{$q_1$};
		\node[state]				(1) [right of=0]		{$q_2$};
		\node[state]				(2) [right of=1]		{$q_3$};
		\node[state]				(3) [right of=2]		{$q_4$};
		\node[state]				(4) [right of=3] 		{$q_5$};
		\node[state]				(6) [below right of=5]	{$q_6$};
		\node[state]				(8) [below right of=4] 	{$q_8$};
		\node[state]				(7) [below left of=8]	{$q_7$};
		\node[state,accepting]		(9) [right of= 8]		{$q_9$};
		
		\path
		(0)		edge 						node {$\epsilon$}	(1)
		(1)		edge [below]				node {$\mathbf{0}$}	(2)
		(2)		edge [below]				node {$\mathbf{1}$}	(3)
		(3)		edge [bend right=50,above]	node {$\epsilon$}	(1)
		(0)		edge [bend right=40,below]	node {$\epsilon$}	(4)
		(3)		edge 						node {$\epsilon$}	(4)
		(4)		edge 						node {$\epsilon$}	(8)
		(5)		edge 						node {$\epsilon$}	(0)
		(5)		edge [below left]			node {$\epsilon$}	(6)
		(6)		edge [below]				node {$\mathbf{1}$}	(7)
		(7)		edge [below right]			node {$\epsilon$}	(8)
		(8)		edge 						node {$\mathbf{0}$}	(9)
		;
		\end{tikzpicture}
		\end{center}
		
		\pagebreak
		
		Nun erstellen wir aus dem oben dargestellten $\epsilon$-NEA $A$ einen EA $B$. Zuerst fassen wir $\textsc{eclose}(q)$ für $q \in \{q_0,q_1,q_2,q_3,q_4,q_5,q_6,q_7,q_8,q_9\}$ in einer Tabelle zusammen, weil wir sie häufig brauchen werden.
		
		\begin{center}
		\begin{tabular}{c|l}
		$q$ & $\textsc{eclose}(q)$ \\ 
		 \hline
		$q_0$ & $\{q_0,q_1,q_2,q_5,q_6,q_8\}$ \\ 
		$q_1$ & $\{q_1,q_2,q_5,q_8\}$ \\ 
		$q_2$ & $\{q_2\}$ \\ 
		$q_3$ & $\{q_3\}$ \\ 
		$q_4$ & $\{q_2,q_4,q_5,q_8\}$ \\ 
		$q_5$ & $\{q_5,q_8\}$ \\ 
		$q_6$ & $\{q_6\}$ \\ 
		$q_7$ & $\{q_7,q_8\}$ \\ 
		$q_8$ & $\{q_8\}$ \\ 
		$q_9$ & $\{q_9\}$ \\ 
		\end{tabular}
		\end{center}
		
		Der Startzustand in $A$ ist $q_0$, also ist der Startzustand von $B$ $\textsc{eclose}(q_0)=\{q_0,q_1,q_2,q_5,q_6,q_8\}$. Jetzt suchen wir die Nachfolger der Elemente in $\{q_0,q_1,q_2,q_5,q_6,q_8\}$ für die Eingaben 0 und 1. Wir berechnen dafür zwei Übergänge und gehen dabei direkt rekursiv weiter:
		
		\begin{enumerate}
			\item
			$\delta_B(\{q_0,q_1,q_2,q_5,q_6,q_8\},0)$.
			Die beiden Zustände $q_2,q_8 \in \{q_0,q_1,q_2,q_5,q_6,q_8\}$ gehen auf Eingabe 0 einen Zustand weiter:
			\begin{enumerate}[\quad]
				\item
				$\delta_B(q_2,0)=\{q_3\}$
				\item
				$\delta_B(q_8,0)=\{q_9\}$
			\end{enumerate}
			Also $\epsilon$-schliessen wir $q_3$ und $q_9$:
			\begin{enumerate}[\quad]
				\item
				$\textsc{eclose}(q_3) = \{q_3\}$
				\item
				$\textsc{eclose}(q_9) = \{q_9\}$
			\end{enumerate}
			Das ergibt einen neuen Zustand $\{q_3,q_9\}$.
			\begin{enumerate}[\quad]
				\item
				Der einzige Zustand in $\{q_3,q_9\}$, der ausgehende Pfeile mit 0 oder 1 hat, ist $q_3$:
				\begin{enumerate}[\quad]
					\item
					$\delta_B(q_3,1)=\{q_4\}$
				\end{enumerate}
				Also $\epsilon$-schliessen wir $q_4$:
				\begin{enumerate}[\quad]
					\item
					$\textsc{eclose}(q_4) = \{q_2,q_5,q_8\}$.
				\end{enumerate}
				Das ergibt einen neuen Zustand $\{q_2,q_5,q_8\}$.
				\begin{enumerate}[\quad]
					\item
					Die Zustände $q_2$ und $q_8\}$ haben ausgehende Pfeile mit 0:
					\begin{enumerate}[\quad]
						\item
						$\delta_B(q_2,0)=\{q_3\}$
						\item
						$\delta_B(q_8,0)=\{q_9\}$
					\end{enumerate}
					Also $\epsilon$-schliessen wir $q_3$ und $\{q_9\}$:
					\begin{enumerate}[\quad]
						\item
						$\textsc{eclose}(q_3) = \{q_3\}$.
						\item
						$\textsc{eclose}(q_9) = \{q_9\}$.
					\end{enumerate}
					Der Zustand $\{q_3,q_9\}$ existiert bereits und damit ist der Pfad zu Ende.
				\end{enumerate}
			\end{enumerate}
			\item
			$\delta_B(\{q_0,q_1,q_2,q_5,q_6,q_8\},1)$.
			Die beiden Zustände $q_3 \notin \{q_0,q_1,q_2,q_5,q_6,q_8\}$ und $q_6 \in \{q_0,q_1,q_2,q_5,q_6,q_8\}$ gehen auf Eingabe 1 einen Zustand weiter:
			\begin{enumerate}[\quad]
				\item
				$\delta_B(q_6,1)=\{q_7\}$
			\end{enumerate}
			Also $\epsilon$-schliessen wir $q_7$:
			\begin{enumerate}[\quad]
				\item
				$\textsc{eclose}(q_7) = \{q_7,q_8\}$
			\end{enumerate}
			Das ergibt einen neuen Zustand $\{q_7,q_8\}$.
			\begin{enumerate}[\quad]
				\item
				Der einzige Zustand in $\{q_7,q_8\}$, der ausgehende Pfeile mit 0 oder 1 hat, ist $q_8$:
				\begin{enumerate}[\quad]
					\item
					$\delta_B(q_8,0)=\{q_9\}$
				\end{enumerate}
				Also $\epsilon$-schliessen wir $q_9$:
				\begin{enumerate}[\quad]
					\item
					$\textsc{eclose}(q_9) = \{q_9\}$.
				\end{enumerate}
				Das ergibt einen neuen Zustand $\{q_9\}$. Der Pfad ist zu Ende, weil keine ausgehenden Pfeile mit 0 oder 1 mehr vorkommen.
			\end{enumerate}
		\end{enumerate}
		
		Daraus konstruieren wir nun den EA. Wir definieren:
		\begin{enumerate}[\quad]
			\item
			$p_0 := \{q_0,q_1,q_2,q_5,q_6,q_8\}$
			\item
			$p_1 := \{q_3,q_9\}$
			\item
			$p_2 := \{q_2,q_5,q_8\}$
			\item
			$p_3 := \{q_7,q_8\}$
			\item
			$p_4 := \{q_9\}$
		\end{enumerate}
		Die akzeptierten Zustände sind alle, die $q_9$ enthalten, also $p_1$ und $p_4$. Zusätzlich erstellen wir den toten Zustand $\emptyset$, zu dem alle übrigen Pfeile (alle, die wir nicht wie oben konstruiert haben) führen, inkl. eines Loops auf sich selbst. Also erhalten wir den folgenden EA:
		
		\begin{center}
		\begin{tikzpicture}[->,>=stealth',shorten >=1pt,auto,node distance=2.5cm,
		semithick,every node/.style={scale=1}]
		\tikzstyle{every state}=[circle,text=black]
		
		\node[initial,state]			(0)             		{$p_0$};
		\node[state,accepting]			(1) [above right of=0]	{$p_1$};
		\node[state]					(2) [right of=1]		{$p_2$};
		\node[state]					(3) [below right of=0]	{$p_3$};
		\node[state,accepting]			(4) [right of=3]		{$p_4$};
		\node[state]					(5) [above right of=4]	{$\emptyset$};
		
		\path
		(0)		edge 							node {0}		(1)
		(0)		edge [below left]				node {1}		(3)
		(1)		edge [bend right=25]			node {1}		(2)
		(1)		edge [bend right=15]			node {0}		(5)
		(2)		edge [bend right=25,above]		node {0}		(1)
		(2)		edge 							node {1}		(5)
		(3)		edge [below]					node {0}		(4)
		(3)		edge [bend left=15]				node {1}		(5)
		(4)		edge [below right]				node {0,1}		(5)
		(5)		edge [loop right]				node {0,1}		(5)
		;
		\end{tikzpicture}
		\end{center}
	
	\end{homeworkProblem}
	
	\pagebreak
	
	%------------------Aufgabe S5---------------------------------
	\begin{homeworkProblem}
		Wir wenden das Verfahren ,,Converting DFA's to Regular Expressions by Eliminating States'' aus dem Buch \textit{Introduction to Automata
		Theory, Languages, and Computation}, wie in Kapitel 3.2.2 beschrieben, an.
		
		Zuerst wandeln wir den EA
		
		\begin{center}
		\begin{tikzpicture}[->,>=stealth',shorten >=1pt,auto,node distance=2cm,
		semithick,every node/.style={scale=1}]
		\tikzstyle{every state}=[circle,text=black]
		
		\node[initial,state]			(1)             		{$1$};
		\node[state,accepting]			(2) [right of=1]		{$2$};
		\node[state]					(3) [right of=2]		{$3$};
		
		\path
		(1)		edge [loop above]					node {$b$}		(1)
		(1)		edge 								node {$a,c$}	(2)
		(2)		edge [loop above]					node {$a$}		(2)
		(2)		edge [bend left]					node {$b,c$}	(3)
		(3)		edge [loop above]					node {$a,c$}	(3)
		(3)		edge [bend left]					node {$b$}		(2)
		;
		\end{tikzpicture}
		\end{center}
		
		in einen EA mit regulären Ausdrücken als Beschriftungen um:
		
		\begin{center}
		\begin{tikzpicture}[->,>=stealth',shorten >=1pt,auto,node distance=2cm,
		semithick,every node/.style={scale=1}]
		\tikzstyle{every state}=[circle,text=black]
		
		\node[initial,state]			(1)             		{$1$};
		\node[state,accepting]			(2) [right of=1]		{$2$};
		\node[state]					(3) [right of=2]		{$3$};
		
		\path
		(1)		edge [loop above]					node {$\mathbf{b}$}		(1)
		(1)		edge 								node {$\mathbf{a+c}$}	(2)
		(2)		edge [loop above]					node {$\mathbf{a}$}		(2)
		(2)		edge [bend left]					node {$\mathbf{b+c}$}	(3)
		(3)		edge [loop above]					node {$\mathbf{a+c}$}	(3)
		(3)		edge [bend left]					node {$\mathbf{b}$}		(2)
		;
		\end{tikzpicture}
		\end{center}
		
		Der einzige Zustand, den wir eliminieren wollen, ist der Zustand 3. Zustand 3 hat einen Vorgänger, 2, und einen Nachfolger, ebenfalls 2. In den regulären Ausdrücken von Fig. 3.7 im Buch ausgedrückt bedeutet das: $Q_1=\mathbf{b+c}$ (der Pfeil von Zustand 2 zu 3), $P_1=\mathbf{b}$ (der Pfeil von Zustand 3 zu 2), $R_{11}=\mathbf{a}$ (der Pfeil von Zustand 2 zu 2, also der Loop auf Zustand 2) und $S=\mathbf{a+c}$ (der Loop auf Zustand 3).
		
		Der entstehende Pfeil von Zustand 2 zu 2, also ein Loop auf Zustand 2, berechnet sich mit $R_{11}+Q_1S^*P_1$ (entweder gehen wir direkt über $R_{11}$ zu 2 oder über $Q_1S^*P_1$). Es gilt:
		\[
			R_{11}+Q_1S^*P_1 =
			\mathbf{a+(b+c)(a+c)^*b}.
		\]
		Der Ausdruck kann nicht vereinfacht werden. Nun haben wir einen EA mit zwei Zuständen erstellt:
		
		\begin{center}
		\begin{tikzpicture}[->,>=stealth',shorten >=1pt,auto,node distance=3cm,
		semithick,every node/.style={scale=1}]
		\tikzstyle{every state}=[circle,text=black]
		
		\node[initial,state]			(1)             		{$1$};
		\node[state,accepting]			(2) [right of=1]		{$2$};
		
		\path
		(1)		edge [loop above]		node {$\mathbf{b}$}					(1)
		(1)		edge 					node {$\mathbf{a+c}$}				(2)
		(2)		edge [loop above]		node {$\mathbf{a+(b+c)(a+c)^*b}$}	(2)
		;
		\end{tikzpicture}
		\end{center}
		
		Die regulären Ausdrücke des Automaten sind nach dem generischen Zwei-Zustand-Automaten aus Fig. 3.9: $R=\mathbf{b}$, $S=\mathbf{a+c}$, $T=\emptyset$ und $U=\mathbf{a+(b+c)(a+c)^*b}$. Wir berechnen nun
		
		\begin{align*}
			(R+SU^*T)^*SU^* &= \mathbf{\Big(b+(a+c)\big(a+(b+c)(a+c)^*b\big)^*\emptyset\Big)^*(a+c)\big(a+(b+c)(a+c)^*b\big)^*} &\\
			&= \mathbf{(b+\emptyset)^*(a+c)\big(a+(b+c)(a+c)^*b\big)^*} & (\forall \mathbf{k}: \mathbf{k}\emptyset = \emptyset)\\
			&= \mathbf{b^*(a+c)\big(a+(b+c)(a+c)^*b\big)^*} & (\forall \mathbf{k}: \mathbf{k}+\emptyset = \mathbf{k}).
		\end{align*}
		
		Das ergibt den folgenden Ein-Zustand-Automaten:
		
		\begin{center}
		\begin{tikzpicture}[->,>=stealth',shorten >=1pt,auto,node distance=3cm,
		semithick,every node/.style={scale=1}]
		\tikzstyle{every state}=[circle,text=black]
		
		\node[initial,state,accepting]	(1)  		{$1$};
		
		\path
		(1)		edge [loop above]
					node{$\mathbf{b^*(a+c)\big(a+(b+c)(a+c)^*b\big)^*}$}		(1)
		;
		\end{tikzpicture}
		\end{center}
		
		Damit ist der gesuchte reguläre Ausdruck $\mathbf{b^*(a+c)\big(a+(b+c)(a+c)^*b\big)^*}$.
	\end{homeworkProblem}
\end{document}
