%BEGIN_FOLD
%%%%%%%%%%%%%%%%%%%%%%%%%%%%%%%%%%%%%%%%
% Programming/Coding Assignment
% LaTeX Template
%
% Original author:
% Ted Pavlic (http://www.tedpavlic.com)
%
%
% This template uses a Perl script as an example snippet of code, most other
% languages are also usable. Configure them in the "CODE INCLUSION 
% CONFIGURATION" section.
%
%%%%%%%%%%%%%%%%%%%%%%%%%%%%%%%%%%%%%%%%%
%END_FOLD
%----------------------------------------------------------------------------------------
%BEGIN_FOLD	%PACKAGES AND OTHER DOCUMENT CONFIGURATIONS
%----------------------------------------------------------------------------------------

\documentclass{article}
\usepackage[utf8]{inputenc}

\usepackage[ngerman]{babel}
\usepackage{amsmath}
\usepackage{amssymb}
\usepackage{mathtools}
\usepackage{amsfonts}
\usepackage{fancyhdr} % Required for custom headers
\usepackage{lastpage} % Required to determine the last page for the footer
\usepackage{extramarks} % Required for headers and footers
\usepackage[usenames,dvipsnames]{color} % Required for custom colors
\usepackage{graphicx} % Required to insert images
\usepackage{listings} % Required for insertion of code
\usepackage{courier} % Required for the courier font
\usepackage{enumerate} % used for enumerate args
\usepackage{multicol} % columns

\usepackage{pgf} 
\usepackage{tikz}
\usepackage{forest} % treees :D
\usetikzlibrary{arrows,automata} %for FSM
\usepackage{mathtools}
\usepackage{wasysym}

% Custom commands
\DeclareMathOperator{\Space}{Space}
\DeclareMathOperator{\Time}{Time}
\DeclareMathOperator{\Kl}{Kl} %Klassen von Zuständen
\newcommand{\Ebool}{\{0, 1\}}
\newcommand{\irow}[1]{% inline row vector
	\begin{smallmatrix}(#1)\end{smallmatrix}
}
\newcommand{\icol}[1]{% inline column vector
	\left(\begin{smallmatrix}#1\end{smallmatrix}\right)%
}
\DeclarePairedDelimiter{\ceil}{\lceil}{\rceil}
% Shamelessly copied from http://tex.stackexchange.com/questions/43008/absolute-value-symbols
\DeclarePairedDelimiter\abs{\lvert}{\rvert} % nice |x|
\DeclarePairedDelimiter\norm{\lVert}{\rVert} % nice ||x||
% Swap the definition of \abs* and \norm*, so that \abs
% and \norm resizes the size of the brackets, and the 
% starred version does not.
%END_FOLD
%----------------------------------------------------------------------------------------
%BEGIN_FOLD
\makeatletter
\let\oldabs\abs
\def\abs{\@ifstar{\oldabs}{\oldabs*}}
\let\oldnorm\norm
\def\norm{\@ifstar{\oldnorm}{\oldnorm*}}
\makeatother


% Margins
\topmargin=-0.45in
\evensidemargin=0in
\oddsidemargin=0in
\textwidth=6.5in
\textheight=9.0in
\headsep=0.25in

\linespread{1.1} % Line spacing

% Set up the header and footer
\pagestyle{fancy}
\lhead{\hmwkAuthorName} % Top left header
%\chead{\hmwkClass\ (\hmwkClassInstructor\): \hmwkTitle} % Top center head
%\rhead{\firstxmark} % Top right header
\rhead{}
\lfoot{\lastxmark} % Bottom left footer
\cfoot{} % Bottom center footer
\rfoot{Seite\ \thepage\ von\ \protect\pageref{LastPage}} % Bottom right footer
\renewcommand\headrulewidth{0.4pt} % Size of the header rule
\renewcommand\footrulewidth{0.4pt} % Size of the footer rule

\setlength\parindent{0pt} % Removes all indentation from paragraphs

%----------------------------------------------------------------------------------------
%	CODE INCLUSION CONFIGURATION
%----------------------------------------------------------------------------------------

\definecolor{MyDarkGreen}{rgb}{0.0,0.4,0.0} % This is the color used for comments
\lstloadlanguages{Pascal} % Load Pascal syntax for listings, for a list of other languages supported see: ftp://ftp.tex.ac.uk/tex-archive/macros/latex/contrib/listings/listings.pdf
\lstset{language=Perl, % Use Pascal in this example
	frame=single, % Single frame around code
	basicstyle=\small\ttfamily, % Use small true type font
	keywordstyle=[1]\color{Blue}\bf, % Pascal functions bold and blue
	keywordstyle=[2]\color{Purple}, % Pascal function arguments purple
	keywordstyle=[3]\color{Blue}\underbar, % Custom functions underlined and blue
	identifierstyle=, % Nothing special about identifiers                                         
	commentstyle=\usefont{T1}{pcr}{m}{sl}\color{MyDarkGreen}\small, % Comments small dark green courier font
	stringstyle=\color{Purple}, % Strings are purple
	showstringspaces=false, % Don't put marks in string spaces
	tabsize=5, % 5 spaces per tab
	%
	% Put standard Pascal functions not included in the default language here
	morekeywords={rand},
	%
	% Put Pascal function parameters here
	morekeywords=[2]{on, off, interp},
	%
	% Put user defined functions here
	morekeywords=[3]{test},
	%
	morecomment=[l][\color{Blue}]{...}, % Line continuation (...) like blue comment
	numbers=left, % Line numbers on left
	firstnumber=1, % Line numbers start with line 1
	numberstyle=\tiny\color{Blue}, % Line numbers are blue and small
	stepnumber=5 % Line numbers go in steps of 5
}

% Creates a new command to include a perl script, the first parameter is the filename of the script (without .p), the second parameter is the caption
\newcommand{\pascalscript}[2]{
	\begin{itemize}
		\item[]\lstinputlisting[caption=#2,label=#1]{#1.p}
	\end{itemize}
}

%----------------------------------------------------------------------------------------
%	DOCUMENT STRUCTURE COMMANDS
%	Skip this unless you know what you're doing
%----------------------------------------------------------------------------------------

% Header and footer for when a page split occurs within a problem environment
%\newcommand{\enterProblemHeader}[1]{
%\nobreak\extramarks{#1}{#1 continued on next page\ldots}\nobreak
%\nobreak\extramarks{#1 (continued)}{#1 continued on next page\ldots}\nobreak
%}

% Header and footer for when a page split occurs between problem environments
%\newcommand{\exitProblemHeader}[1]{
%\nobreak\extramarks{#1 (continued)}{#1 continued on next page\ldots}\nobreak
%\nobreak\extramarks{#1}{}\nobreak
%}

\setcounter{secnumdepth}{0} % Removes default section numbers
\newcounter{homeworkProblemCounter} % Creates a counter to keep track of the number of problems

\newcommand{\homeworkProblemName}{}
%END_FOLD
\newenvironment{homeworkProblem}[1][Aufgabe \arabic{homeworkProblemCounter}]
%BEGIN_FOLD
{ % Makes a new environment called homeworkProblem which takes 1 argument (custom name) but the default is "Problem #"
	
	\stepcounter{homeworkProblemCounter} % Increase counter for number of problems
	\renewcommand{\homeworkProblemName}{#1} % Assign \homeworkProblemName the name of the problem
	\section{\homeworkProblemName} % Make a section in the document with the custom problem count
	%\enterProblemHeader{\homeworkProblemName} % Header and footer within the environment
}{
%\exitProblemHeader{\homeworkProblemName} % Header and footer after the environment
}

\newcommand{\problemAnswer}[1]{ % Defines the problem answer command with the content as the only argument
	\noindent\framebox[\columnwidth][c]{\begin{minipage}{0.98\columnwidth}#1\end{minipage}} % Makes the box around the problem answer and puts the content inside
}

\newcommand{\homeworkSectionName}{}
\newenvironment{homeworkSection}[1]{ % New environment for sections within homework problems, takes 1 argument - the name of the section
	\renewcommand{\homeworkSectionName}{#1} % Assign \homeworkSectionName to the name of the section from the environment argument
	\subsection{\homeworkSectionName} % Make a subsection with the custom name of the subsection
	%\enterProblemHeader{\homeworkProblemName\ [\homeworkSectionName]} % Header and footer within the environment
}{
%\enterProblemHeader{\homeworkProblemName} % Header and footer after the environment
}
%END_FOLD
%----------------------------------------------------------------------------------------
%BEGIN_FOLD %NAME AND CLASS SECTION
%----------------------------------------------------------------------------------------

\newcommand{\hmwkTitle}{Blatt} % Assignment title
\newcommand{\hmwkDueDate}{13.\ November\ 2015} % Due date
\newcommand{\hmwkClass}{Theoretische Informatik} % Course/class
\newcommand{\hmwkClassInstructor}{} % Teacher/lecturer
\newcommand{\hmwkAuthorName}{Linus Fessler, Markus Hauptner, Philipp Schimmelfennig} % Your name
\newcommand{\hmwkNumber}{7}
\newcommand{\hmwkAssiInfo}{Assistent: Sacha Krug, CHN D 42}
%END_FOLD
%----------------------------------------------------------------------------------------
%BEGIN_FOLD
%----------------------------------------------------------------------------------------
%	TITLE PAGE
%----------------------------------------------------------------------------------------

\title{
	\vspace{2in}
	\textmd{\textbf{\hmwkClass:\ \hmwkTitle\ \hmwkNumber}}\\
	\normalsize\vspace{0.1in}\small{Abgabe\ bis\ \hmwkDueDate}
	\\\hmwkAssiInfo
	\\
	\vspace{0.1in}\large{\textit{\hmwkClassInstructor}
		\vspace{3in}
	}}
	\author{\textbf{\hmwkAuthorName}}
	\date{} % Insert date here if you want it to appear below your name
	
	%----------------------------------------------------------------------------------------
	%END_FOLD
	\begin{document}
		%BEGIN_FOLD
		\maketitle
		
		%----------------------------------------------------------------------------------------
		%	TABLE OF CONTENTS
		%----------------------------------------------------------------------------------------
		
		%\setcounter{tocdepth}{1} % Uncomment this line if you don't want subsections listed in the ToC
		%END_FOLD
		%----------------------------------------------------------------------------------------
		\addtocounter{homeworkProblemCounter}{18}
		%----------------------------------------------------------------------------------------
		\newpage
		%\tableofcontents
		%\newpage
		
		%----------------------------------------------------------------------------------------
		%	Aufgabe S1
		%----------------------------------------------------------------------------------------
		
		\begin{homeworkProblem}
			Sei $A$ eine $k$-Band-TM. Wir konstruieren nun nach und nach eine 1-Band-TM $B$. Die Vorgehensweise ist analog zum Beweis von \textit{Lemma 4.2}
			\begin{enumerate}[1.]
				\item
				Zu jedem Band $i$ von A bis auf das Eingabeband fügen wir ein zweites Band $i'$ direkt darunter ein. Dieses soll nur die Symbole $\{\cent,\,\textvisiblespace,\,\uparrow\}$ beinhalten und $\uparrow$ soll nur einmal auf diesem Band vorkommen. Der Pfeil indiziert die Position des Kopfes, der in $A$ auf Band $i$ zeigt. Die Anzahl an Arbeitsbändern verdoppelt sich.
				
				\item
				Jetzt fassen wir sämtliche Arbeitsbänder $i$ und $i'$, $i\in\{1, 2, ..., k\}$ zu einem Band zusammen. Der $j$-te Eintrag auf den Arbeitsband von $B$ muss also die Infos über dem $j$-ten Eintrag sämtlicher Arbeitsbänder $i$ und $i'$ beinhalten. Das Eingabeband von $A$ lassen wir unangetastet.
				Dadurch vergrößert sich unser Arbeitsalphabet.
				\[
					\Gamma_B = \underbrace{\Sigma_A \cup \{\cent,\,\$\}}_{(1)} \;\;\underbrace{\cup}_{(2)}\;\;
					\Bigl(
						\;\underbrace{\bigl\{
							\Sigma_A \cup \{\cent,\,\textvisiblespace\}
						\bigr\}}_{(3)}
						\times 
						\underbrace{\bigl\{\textvisiblespace, \uparrow\bigr\}}_{(4)}\;
					\Bigr)^k
				\]
				\begin{enumerate}[(1)]
					\item Alle Zeichen, die im Eingabealphabet stehen können.
					\item wir lesen entweder ein Zeichen auf dem Eingabeband oder auf dem Arbeitsband.
					\item Zeichen auf dem $i$-ten Arbeitsband von $A$.
					\item Zeichen auf dem $i'$-ten Arbeitsband (siehe oben).
				\end{enumerate}
				
				\item
				\begin{enumerate}[(a)]
					\item $B$ liest einmal den Inhalt des Eingabebandes von links nach rechts, bis alle $k$ Kopfpositionen von $A$ gefunden wurden und speichert dabei in ihrem Zustand die $k$ Symbole, die sie bei den $k$ Köpfen von $A$ gelesen hat. (Das sind die Symbole, die auf der Spur über `$\uparrow$' stehen.)
					Die Anzahl Zustände in $B$ ist also grösser als die in $A$.
						
					\item Jetzt kennt $B$ das ganze Argument ($B$ hat den zu $A$ passenden Zustand.) der Transitfuntionen von $A$ und kann entsprechend die Köpfe bewegen (= Symbole auf der $2i$-ten Spur ersetzen).
					Diese Änderung kann $B$ in einem Lauf von rechts nach links ausführen. 
				\end{enumerate}
			\end{enumerate}
			$\Space_B(n)$ ist so groß, wie die Längste Konfigutation von allen Berechnungsschritten von $B$ auf allen Wörtern über $\Sigma$ der Länge $n$ auf dem Arbeitsband. Das Arbeitsband hat nach der genannten Konstruktion die beschriebene Länge des Maximums der beschriebenen Längen aller $k$  Arbeitsbänder von $A$, für alle $n$. Daher folgt $\Space_B(n) \leq \Space_A(n)\quad$, für alle $n$.\\
			
			In jedem Simulationsschritt muss das Arbeitsband ein Mal von links nach rechts und wieder zurück gelesen werden. $\Rightarrow$ Für den $i$-ten Schritt sind $\Space_B(C_i)$ Schritte notwendig.\\
			
			$\Time_B(n)$ im Verhältnis zu $\Time_A(n)$:
			\[
				\Time_B(n) \;\leq\; \Time_A(n) \cdot \Space_B(n) \;\leq\; \Time_A(n) \cdot \Space_A(n)
			\]
		\end{homeworkProblem}
		\begin{homeworkProblem}
			\begin{enumerate}[(a)]
				\item
				$e(n)=2^n$
				
				Wir konstruieren eine 2-Band Turingmaschine $M$, wobei \textit{Band 0} das Eingabeband und \textit{Bänder 1--2} die Arbeitsbänder sind. $M$ bekommt als Eingabe das Wort $0^n$ auf \textit{Band 0}. 
				Zu Beginn schreibt $M$ eine $0$ auf \textit{Band 1}.
				Solange der Lesekopf des Eingabebandes nicht \$ liest:
				\begin{enumerate}[1.]
					\item
					Gehe auf \textit{Band 1} nach links bis $\cent$. 
					\item
					Gehe auf \textit{Band 2} nach links bis $\cent$
					\item
					Lies Zeichen auf \textit{Band 1}. Schreibe für jede gelesene $0$ auf \textit{Band 1} $00$ auf \textit{Band 2}. Für ein $\textvisiblespace$ schreibe ein $\textvisiblespace$.
					\item
					Gehe auf beiden Bändern nach links und kopiere Inhalt von \textit{Band 2} auf \textit{Band 1} einschließlich bis Zeichen $\textvisiblespace$.
					\item
					Rücke mit Lesekopf nach rechts.
				\end{enumerate}
				Das Ergebnis steht dann auf \textit{Band 2} bis zum ersten $\textvisiblespace$.
				
				Auf diese Art generieren wir $2^n$ Nullen.
				Für $n$ Nullen der Eingabe lesen wir pro Schritt $2^i$ Nullen. Das Schreiben geschieht jeweils in $\mathcal{O}(1)$.
				\[
				\sum_{i=1}^{n}2^i = 2^{n+1} - 2 \in \mathcal{O}(2^n) 
				\]
				Folglich ist $e(n)$ zeitkonstruierbar.\\
				
				\item 
				$f(n)=fib_n$
				
				Wir konstruieren eine 3-Band Turingmaschine $M$, wobei \textit{Band 0} das Eingabeband und \textit{Bänder 1--3} die Arbeitsbänder sind. $M$ bekommt als Eingabe das Wort $w=0^n$ auf \textit{Band 0}.
				Wir unterscheiden drei Eingaben $w$.
				\begin{enumerate}[{Fall} 1:]
					\item $w=\lambda$.\\
					In diesem Fall ist $n=0$. $M$ schreibt $0^{fib_0}=\lambda$ auf
					\item $w=0$.\\
					In diesem Fall ist $n=1$. $M$ schreibt $0^{fib_1}=0$ auf \textit{Band 1} und hält.
					\item $|w|=n, n \geq 2$.
					Der Lesekopf auf \textit{Band 0} liegt auf der zweiten 0 und $M$ liest auf \textit{Band 0} von links nach rechts. Für die erste gelesene 0 schreibt $M$ eine 0 auf \textit{Band 2}. Für jede weitere gelesene 0 führt $M$  Schritte 1.--3. aus, bis \$ gelesen wird. Dann ist auf \textit{Band 1} das Ergebnis $0^{fib_n}$ (und auf \textit{Band 2} $0^{fib_{n-1}}$ und auf \textit{Band 3} $0^{fib_{n-2}}$).
					\begin{enumerate}[1.]
						\item
						$M$ überschreibt \textit{Band 3} mit dem Inhalt von \textit{Band 2} ($fib_{i-2} \leftarrow fib_{i-1}$).
						\item
						$M$ überschreibt \textit{Band 2} mit dem Inhalt von \textit{Band 1} ($fib_{i-1} \leftarrow fib_{i}$).
						\item
						$M$ schreibt den Inhalt von \textit{Band 3} in \textit{Band 1} (konkateniert also die Nullen auf \textit{Band 1} mit den Nullen von \textit{Band 3}) ($fib_{i} \leftarrow fib_{i-1} + fib_{i-2}$).
					\end{enumerate}
				\end{enumerate}
				In den Fällen 1 und 2 ist die Laufzeit konstant, also ist $f(n) \in \mathcal{O}(1)$. Im Fall 3 schreiben wir bei der $i-ten$ Ausführung, $i \leq n-2$ (weil wir auf der zweiten 0 starten), $fib_{i-2}$ Nullen auf \textit{Band 3}, $fib_{i-1}$ Nullen auf \textit{Band 2} und $fib_{i-2}$ Nullen auf \textit{Band 1}, also pro Schritt $2fib_{i-2}+fib_{i-1} \leq 3fib_{i-1}$ Nullen. Insgesamt ergibt das also
				\begin{align*}
				3\sum_{i=2}^{n}fib_{i-1} & = 3\sum_{i=1}^{n-1}fib_{i} & (\text{Indexverschiebung}) \\
				& = 3\sum_{i=0}^{n-1}fib_{i} & (fib_0 = 0) \\
				& = 3(fib_{n+1}-1). & (\sum_{i=0}^{n}fib_{i} = fib_{n+2}-1) \\
				& = 3(fib_n+fib_{n-1}-1) & (\text{Definition } fib_n) \\
				& \leq 3(2fib_n-1)
				\end{align*}
				$3(2fib_n-1) = 6fib_n-2 \in \mathcal{O}(fib_n)$. Damit ist $f(n)$ zeitkonstruierbar. 
			\end{enumerate}
		\end{homeworkProblem}
		
		\begin{homeworkProblem}
			Wir wissen: $\; f:\mathbb{N}\rightarrow\mathbb{N},\;\; g:\mathbb{N}\rightarrow\mathbb{N}\;\;$ und $f$ und $g$ sind beide platzkonstruierbar.\\
			$\Rightarrow$ Es gibt 1-Band-Turingmaschinen $F$ und $G$, so dass  $\begin{array}{ll}
			\Space_F(n_1)\leq f(n_1)\\
			\Space_G(n_2)\leq g(n_2)\\
			\end{array} \quad\forall n_1, n_2 \in \mathbb{N}$\\
			und für jede Eingabe 
			$\begin{array}{ll}
			0^{n_1} \text{ generiert $F$ das Wort } 0^{f(n_1)}\\
			0^{n_2} \text{ generiert $G$ das Wort } 0^{g(n_2)}
			\end{array}$
			auf ihrem Arbeitsband und hält in Zustand $q_{accept}$.\\\\
			Wir konstruieren eine 5-Band-Turingmaschine $H$. $H$ bekommt als Inpt $0^n$ auf sein Eingabeband. $H$ kopiert die Eingabe auf \textit{Band 2} und auf \textit{Band 4} und simuliert F, dann G. Dabei sind \textit{Band 2, 3} das Eingabe- und Arbeitsband von F und \textit{Band 4, 5} Eingabe- und Arbeitsband von G.\\
			Auf \textit{Band 3} steht nun $0^{f(n)}$ und auf \textit{Band 5} steht $0^{g(n)}$.
			$H$ geht nach an den Anfang von \textit{Band 3} und geht für jede gelesene $0$ eins nach rechts und hängt den gesamten Inhalt von \textit{Band 5} an \textit{Band 1} an. Hat $H$ alle $0$en auf \textit{Band 3} gelesen steht auf \textit{Band 1} nun $0^{f(n)\cdot g(n)}$. $H$ akzeptiert.\\
			Die längte Konfiguration über alle Bänder und Schritte hat $H$ am Ende, wenn das Ergebnis steht. Damit ist 
			\[
			\Space_H = f(n)\cdot g(n) =: h(n) \quad \forall n\in\mathbb{N} \quad \textit{(nach Def. 6.2)} 
			\]
			und $H$ hält immer in $q_{accept}$.\\ Nach \textit{Lemma 6.1} gibt es eine äquivalente 1-Band-Turingmaschine $H'$ mit $\Space_{H'} \leq \Space_H \leq h(n)$. Folglich ist $h(n)$ platzkonstruierbar.
		\end{homeworkProblem}
		
	\end{document}
	