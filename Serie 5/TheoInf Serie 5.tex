%BEGIN_FOLD
%%%%%%%%%%%%%%%%%%%%%%%%%%%%%%%%%%%%%%%%
% Programming/Coding Assignment
% LaTeX Template
%
% Original author:
% Ted Pavlic (http://www.tedpavlic.com)
%
%
% This template uses a Perl script as an example snippet of code, most other
% languages are also usable. Configure them in the "CODE INCLUSION 
% CONFIGURATION" section.
%
%%%%%%%%%%%%%%%%%%%%%%%%%%%%%%%%%%%%%%%%%
%END_FOLD
%----------------------------------------------------------------------------------------
%BEGIN_FOLD	%PACKAGES AND OTHER DOCUMENT CONFIGURATIONS
%----------------------------------------------------------------------------------------

\documentclass{article}
\usepackage[utf8]{inputenc}

\usepackage[ngerman]{babel}
\usepackage{amsmath}
\usepackage{amssymb}
\usepackage{mathtools}
\usepackage{amsfonts}
\usepackage{fancyhdr} % Required for custom headers
\usepackage{lastpage} % Required to determine the last page for the footer
\usepackage{extramarks} % Required for headers and footers
\usepackage[usenames,dvipsnames]{color} % Required for custom colors
\usepackage{graphicx} % Required to insert images
\usepackage{listings} % Required for insertion of code
\usepackage{courier} % Required for the courier font
\usepackage{enumerate} % used for enumerate args
\usepackage{multicol} % columns

\usepackage{pgf} 
\usepackage{tikz}
\usepackage{forest} % treees :D
\usetikzlibrary{arrows,automata} %for FSM
\usetikzlibrary{positioning}
\usepackage{mathtools}

% Custom commands
\DeclareMathOperator{\Kl}{Kl} %Klassen von Zuständen
\newcommand{\Ebool}{\{0, 1\}}
\DeclarePairedDelimiter{\set}{\{}{\}}
\DeclarePairedDelimiter{\ceil}{\lceil}{\rceil}
% Shamelessly copied from http://tex.stackexchange.com/questions/43008/absolute-value-symbols
\DeclarePairedDelimiter\abs{\lvert}{\rvert} % nice |x|
\DeclarePairedDelimiter\norm{\lVert}{\rVert} % nice ||x||
% Swap the definition of \abs* and \norm*, so that \abs
% and \norm resizes the size of the brackets, and the 
% starred version does not.
%END_FOLD
%----------------------------------------------------------------------------------------
%BEGIN_FOLD
\makeatletter
\let\oldabs\abs
\def\abs{\@ifstar{\oldabs}{\oldabs*}}
\let\oldnorm\norm
\def\norm{\@ifstar{\oldnorm}{\oldnorm*}}
\makeatother


% Margins
\topmargin=-0.45in
\evensidemargin=0in
\oddsidemargin=0in
\textwidth=6.5in
\textheight=9.0in
\headsep=0.25in

\linespread{1.1} % Line spacing

% Set up the header and footer
\pagestyle{fancy}
\lhead{\hmwkAuthorName} % Top left header
%\chead{\hmwkClass\ (\hmwkClassInstructor\): \hmwkTitle} % Top center head
%\rhead{\firstxmark} % Top right header
\rhead{}
\lfoot{\lastxmark} % Bottom left footer
\cfoot{} % Bottom center footer
\rfoot{Seite\ \thepage\ von\ \protect\pageref{LastPage}} % Bottom right footer
\renewcommand\headrulewidth{0.4pt} % Size of the header rule
\renewcommand\footrulewidth{0.4pt} % Size of the footer rule

\setlength\parindent{0pt} % Removes all indentation from paragraphs

%----------------------------------------------------------------------------------------
%	CODE INCLUSION CONFIGURATION
%----------------------------------------------------------------------------------------

\definecolor{MyDarkGreen}{rgb}{0.0,0.4,0.0} % This is the color used for comments
\lstloadlanguages{Pascal} % Load Pascal syntax for listings, for a list of other languages supported see: ftp://ftp.tex.ac.uk/tex-archive/macros/latex/contrib/listings/listings.pdf
\lstset{language=Perl, % Use Pascal in this example
	frame=single, % Single frame around code
	basicstyle=\small\ttfamily, % Use small true type font
	keywordstyle=[1]\color{Blue}\bf, % Pascal functions bold and blue
	keywordstyle=[2]\color{Purple}, % Pascal function arguments purple
	keywordstyle=[3]\color{Blue}\underbar, % Custom functions underlined and blue
	identifierstyle=, % Nothing special about identifiers                                         
	commentstyle=\usefont{T1}{pcr}{m}{sl}\color{MyDarkGreen}\small, % Comments small dark green courier font
	stringstyle=\color{Purple}, % Strings are purple
	showstringspaces=false, % Don't put marks in string spaces
	tabsize=5, % 5 spaces per tab
	%
	% Put standard Pascal functions not included in the default language here
	morekeywords={rand},
	%
	% Put Pascal function parameters here
	morekeywords=[2]{on, off, interp},
	%
	% Put user defined functions here
	morekeywords=[3]{test},
	%
	morecomment=[l][\color{Blue}]{...}, % Line continuation (...) like blue comment
	numbers=left, % Line numbers on left
	firstnumber=1, % Line numbers start with line 1
	numberstyle=\tiny\color{Blue}, % Line numbers are blue and small
	stepnumber=5 % Line numbers go in steps of 5
}

% Creates a new command to include a perl script, the first parameter is the filename of the script (without .p), the second parameter is the caption
\newcommand{\pascalscript}[2]{
	\begin{itemize}
		\item[]\lstinputlisting[caption=#2,label=#1]{#1.p}
	\end{itemize}
}

%----------------------------------------------------------------------------------------
%	DOCUMENT STRUCTURE COMMANDS
%	Skip this unless you know what you're doing
%----------------------------------------------------------------------------------------

% Header and footer for when a page split occurs within a problem environment
%\newcommand{\enterProblemHeader}[1]{
%\nobreak\extramarks{#1}{#1 continued on next page\ldots}\nobreak
%\nobreak\extramarks{#1 (continued)}{#1 continued on next page\ldots}\nobreak
%}

% Header and footer for when a page split occurs between problem environments
%\newcommand{\exitProblemHeader}[1]{
%\nobreak\extramarks{#1 (continued)}{#1 continued on next page\ldots}\nobreak
%\nobreak\extramarks{#1}{}\nobreak
%}

\setcounter{secnumdepth}{0} % Removes default section numbers
\newcounter{homeworkProblemCounter} % Creates a counter to keep track of the number of problems

\newcommand{\homeworkProblemName}{}
%END_FOLD
\newenvironment{homeworkProblem}[1][Aufgabe S\arabic{homeworkProblemCounter}]
%BEGIN_FOLD
{ % Makes a new environment called homeworkProblem which takes 1 argument (custom name) but the default is "Problem #"
	
	\stepcounter{homeworkProblemCounter} % Increase counter for number of problems
	\renewcommand{\homeworkProblemName}{#1} % Assign \homeworkProblemName the name of the problem
	\section{\homeworkProblemName} % Make a section in the document with the custom problem count
	%\enterProblemHeader{\homeworkProblemName} % Header and footer within the environment
}{
%\exitProblemHeader{\homeworkProblemName} % Header and footer after the environment
}

\newcommand{\problemAnswer}[1]{ % Defines the problem answer command with the content as the only argument
	\noindent\framebox[\columnwidth][c]{\begin{minipage}{0.98\columnwidth}#1\end{minipage}} % Makes the box around the problem answer and puts the content inside
}

\newcommand{\homeworkSectionName}{}
\newenvironment{homeworkSection}[1]{ % New environment for sections within homework problems, takes 1 argument - the name of the section
	\renewcommand{\homeworkSectionName}{#1} % Assign \homeworkSectionName to the name of the section from the environment argument
	\subsection{\homeworkSectionName} % Make a subsection with the custom name of the subsection
	%\enterProblemHeader{\homeworkProblemName\ [\homeworkSectionName]} % Header and footer within the environment
}{
%\enterProblemHeader{\homeworkProblemName} % Header and footer after the environment
}
%END_FOLD
%----------------------------------------------------------------------------------------
%BEGIN_FOLD %NAME AND CLASS SECTION
%----------------------------------------------------------------------------------------

\newcommand{\hmwkTitle}{Selbststudium - Blatt} % Assignment title
\newcommand{\hmwkDueDate}{23.\ Oktober\ 2015} % Due date
\newcommand{\hmwkClass}{Theoretische Informatik} % Course/class
\newcommand{\hmwkClassInstructor}{} % Teacher/lecturer
\newcommand{\hmwkAuthorName}{Linus Fessler, Markus Hauptner, Philipp Schimmelfennig} % Your name
\newcommand{\hmwkNumber}{1}
\newcommand{\hmwkAssiInfo}{Assistent: Sascha Krug, CHN D 42}
%END_FOLD
%----------------------------------------------------------------------------------------
%BEGIN_FOLD
%----------------------------------------------------------------------------------------
%	TITLE PAGE
%----------------------------------------------------------------------------------------

\title{
	\vspace{2in}
	\textmd{\textbf{\hmwkClass:\ \hmwkTitle\ \hmwkNumber}}\\
	\normalsize\vspace{0.1in}\small{Abgabe\ bis\ \hmwkDueDate}
	\\\hmwkAssiInfo
	\\
	\vspace{0.1in}\large{\textit{\hmwkClassInstructor}
		\vspace{3in}
	}}
	\author{\textbf{\hmwkAuthorName}}
	\date{} % Insert date here if you want it to appear below your name
	
	%----------------------------------------------------------------------------------------
	%END_FOLD
	\begin{document}
		%BEGIN_FOLD
		\maketitle
		
		%----------------------------------------------------------------------------------------
		%	TABLE OF CONTENTS
		%----------------------------------------------------------------------------------------
		
		%\setcounter{tocdepth}{1} % Uncomment this line if you don't want subsections listed in the ToC
		%END_FOLD
		%----------------------------------------------------------------------------------------
		\addtocounter{homeworkProblemCounter}{0}
		%----------------------------------------------------------------------------------------
		\newpage
		%\tableofcontents
		%\newpage
		
		%------------------Aufgabe S1---------------------------------
		
		\begin{homeworkProblem}
			\begin{enumerate}[(a)]
				\item
				Vermutung: L(G$_1)=\{avawa \mid v \in \{b\}^+, w \in \{ab\}^*\}$
				
				Beweis: Die Grammatik beginnt im Startsymbol $S$, für das es nur eine Ableitungsregel gibt: $S \rightarrow aXaYa$. Da das Wort $aXaYa$ immer noch die Nichtterminale $X$ und $Y$ enthält, müssen wir weiter ableiten. Es gibt in $P_1$ keine Ableitungsregel $QaR \rightarrow_{G_1} S$ für beliebige $Q,R,S \in (\Sigma_N \cup \Sigma_T)^*$, womit jedes Wort in der von der Grammatik erzeugten Sprache die Form $avawa$ aufweisen muss, für noch zu bestimmende $v$ und $u$.
				
				Im Folgenden beweisen wir, dass $v \in \{b\}^+$ und $w \in \{ab\}^*$.
				\begin{enumerate}[1.]
					\item
					$v \in \{b\}^+$: Wir haben $aXaYa$ gegeben. Die Ableitungsregeln, die X betreffen, sind $\{X \rightarrow bX, X \rightarrow b\}$. Wir zeigen mit Induktion, dass man mit $X$ und den beiden Ableitungsregel für $X$ ausschliesslich die Wörter $v \in \{b\}^+ = \{b^i \mid i \in \mathbb{N}\}$ generieren kann. Sei $n$ die Anzahl angewandter Ableitungsregeln.\\
					\begin{enumerate}[(i)]
						\item
						\textit{Induktionsanfang:}\\
						Sei $n=1$. Man kann mit einer Ableitung aus $X$ entweder das Wort $bX$ oder $b$ herleiten (nicht aber $\lambda$).\\
						
						\item
						\textit{Induktionsschritt:}\\
						$n \rightarrow n+1$: Es gibt zwei Fälle:\\
						Das Wort endet auf $X$: Man kann wieder  $bX$ oder $b$ herleiten.\\
						Das Wort endet auf $b$: Das Wort kann nicht mehr weiter abgeleitet werden und man erhält nach dem $n$-ten Ableitungsschritt das Wort $b^n$.\\
					\end{enumerate}
					
					\item
					$w \in \{ab\}^*$: Wir haben $aXaYa$ gegeben. Die Ableitungsregeln, die Y betreffen, sind $\{Y \rightarrow abY, Y \rightarrow ab, Y \rightarrow \lambda\}$. Mit Induktion kann man analog zu 1. beweisen, dass man mit $Y$ und den drei Ableitungsregel für $Y$ ausschliesslich die Wörter $w \in \{ab\}^* = \{(ab)^i \mid i \in \mathbb{N}_0\}$ generieren kann. Der einzige Unterschied ist, dass man durch die dritte Ableitungsregel $Y \rightarrow \lambda$ auch das leere Wort herleiten kann.
				\end{enumerate}
				
				\item
				Man beginnt in $S$. Für $S$ gibt es nur eine Ableitungsregel $S \rightarrow 0X0$. Wir gehen also von $0X0$ aus. Für $X$ gibt es nun die drei Ableitungsregeln $\{X \rightarrow AX, X \rightarrow BX, X \rightarrow Y\}$. Nun können wir beliebig viele $A$ oder $B$ vor das $X$ schreiben, bevor wir $X$ durch $Y$ ersetzen.
				
				Damit erhalten wir das Wort $0WY0$ für $W \in \{A,B\}^*$. 
				
				Jetzt können wir sukzessive die Ableitungsregeln $\{AY \rightarrow Y0, BY \rightarrow Y1\}$ anwenden, bis keine $A$ oder $B$ mehr im Wort enthalten sind.
				
				Dann können wir noch einmal die letzte Ableitungsregel $\{0Y \rightarrow 0\}$ anwenden.
				
				So entsteht das Wort $0w0$ für $w \in \{0,1\}^*$, denn die Anwendung von $\{AY \rightarrow Y0, BY \rightarrow Y1\}$ bewirkt schlussendlich, dass jedes $A$ durch eine 0 und jedes $B$ durch eine 1 ersetzt wird und $\{0Y \rightarrow 0\}$ löscht das $Y$ am Ende.
				
				Eine äquivalente reguläre Grammatik ist $G_2' = (\{S, X\}, \{0, 1\}, P_2', S)$ mit
				\[P_2' = \{S \rightarrow 0X0, X \rightarrow \lambda, X \rightarrow 0, X \rightarrow 1, X \rightarrow X0, X \rightarrow X1\}.\]
				Der Entwurf ist praktisch selbsterklärend. Wir starten in $S$, gehen über zu $0X0$ und können dann wählen, ob wir $X$ zu $\lambda$, 0 oder 1 ableiten oder 0 bzw. 1 schreiben und danach weiter Buchstaben aus $\{\lambda,0,1\}$ hinzufügen wollen oder nicht (indem wir Regel $X \rightarrow X0$ oder $X \rightarrow X1$ anwenden). Damit generiert die Grammatik $G_2'$ auch die Wörter $0w0$ für $w \in \{0,1\}^*$.
			\end{enumerate}
		\end{homeworkProblem}
		
		%------------------Aufgabe S2---------------------------------
		\begin{homeworkProblem}
			Die Sprache, die von $G$ erzeugt wird, ist
			\[\{vwaaaxy \mid v=b^{2k+1} \text{ für } k \in \mathbb{N}_0, \quad w,y \in \{a\}^*, \quad x \in \{bb\}^*\}.\]
			Eine äquivalente reguläre Grammatik ist also $G' = (\{S, A, B, C, D, E, F\}, \{a, b\}, P', S)$ mit
			\begin{align*}
				P' = &\{S \rightarrow bA, A \rightarrow bS, A \rightarrow aA, A \rightarrow aB, B \rightarrow aC, C \rightarrow aD, \\ & D \rightarrow bE, E \rightarrow bD, D \rightarrow \lambda, D \rightarrow aF, F \rightarrow aF, F \rightarrow \lambda\}.
			\end{align*}
			Der NEA, der die Sprache $L(G)$ erzeugt, ist damit:
			
			\begin{center}
			\begin{tikzpicture}[->,>=stealth',shorten >=1pt,auto,node distance=2.8cm,
			semithick,every node/.style={scale=1}]
			\tikzstyle{every state}=[circle,text=black]
			
			\node[initial,state]			(S)             		{$S$};
			\node[state]					(A) [right of=S]		{$A$};
			\node[state]					(B) [right of=A]		{$B$};
			\node[state]					(C) [right of=B]		{$C$};
			\node[state,accepting]			(D) [right of=C]		{$D$};
			\node[state]					(E) [below of=D]		{$E$};
			\node[state,accepting]			(F) [right of=D]		{$F$};
			
			\path	(S)		edge [bend left]					node {b}	(A)
					(A)		edge [bend left]					node {b}	(S)
					(A)		edge [loop above]					node {a}	(A)
					(A)		edge 								node {a}	(B)
					(B)		edge 								node {a}	(C)
					(C)		edge 								node {a}	(D)
					(D)		edge [bend left]					node {b}	(E)
					(E)		edge [bend left]					node {b}	(D)
					(D)		edge								node {a}	(F)
					(F)		edge [loop above]					node {a}	(F)
			;
			\end{tikzpicture}
			\end{center}
		\end{homeworkProblem}
		
		%------------------Aufgabe S3---------------------------------
		\begin{homeworkProblem}
			\begin{enumerate}[(a)]
				\item
				Eine reguläre Grammatik für die Sprache $L_1$ ist $G_1 = (\{S, A, B, C, D, E, F\}, \{a, b\}, P_1, S)$ mit
				\begin{align*}
					P_1 = &\{S \rightarrow aA, S \rightarrow bA, A \rightarrow aB, A \rightarrow bB, B \rightarrow aC, B \rightarrow bC, C \rightarrow aA, C \rightarrow bA, B \rightarrow \lambda, \\ & S \rightarrow aD, S \rightarrow bD, S \rightarrow aE, D \rightarrow aD, D \rightarrow bD, D \rightarrow aE, E \rightarrow aF, F \rightarrow \lambda\}.
				\end{align*}
				Um den Entwurf zu begründen ist es einfacher, über den äquivalenten NEA zu argumentieren, der sich leicht von den Regeln ablesen lässt:

				\begin{center}
				\begin{tikzpicture}[->,>=stealth',shorten >=1pt,auto,node distance=2.8cm,
				semithick,every node/.style={scale=1}]
				\tikzstyle{every state}=[circle,text=black]
				
				\node[initial,state]			(S)             		{$S$};
				\node[state,accepting]			(A) [above right of=S]	{$A$};
				\node[state]					(B) [right of=A]		{$B$};
				\node[state]					(C) [right of=B]		{$C$};
				\node[state]					(D) [below right of=S]	{$D$};
				\node[state]					(E) [right of=D]		{$E$};
				\node[state,accepting]			(F) [right of=E]		{$F$};
				
				\path
				(S)		edge 								node {a,b}	(A)
				(S)		edge [below left]					node {a,b}	(D)
				(S)		edge  								node {a}	(E)
				(A)		edge 								node {a,b}	(B)
				(B)		edge 								node {a,b}	(C)
				(C)		edge [bend left]					node {a,b}	(A)
				(D)		edge [loop below]					node {a,b}	(D)
				(D)		edge								node {a}	(E)
				(E)		edge 								node {a}	(F)
				;
				\end{tikzpicture}
				\end{center}
				
				Der NEA wählt zuerst im Startknoten $S$, ob er die Bedingung $(|x|_a + |x|_b) \text{ mod } 3 = 1$ überprüfen will (Knoten $A,B,C$) oder die Bedingung $x$ endet mit $aa$ (Knoten $D,E,F$). Dadurch wird $L_1$ nichtdeterministisch erzeugt und weil der NEA äquivalent zur Grammatik $G_1$ ist, ist $L_1 = L(G_1)$.
				
				\pagebreak
				
				\item
				Eine reguläre Grammatik für die Sprache $L_2$ ist $G_2 = (\{S, A, B, C, D, E, F, G\}, \{0,1\}, P_2, S)$ mit
				\begin{align*}
					P_2 = &\{S \rightarrow 0S, S \rightarrow 1S, S \rightarrow 1A, A \rightarrow 0B, A \rightarrow 1B, B \rightarrow 0C, B \rightarrow 1C,  C \rightarrow 0D, \\ & C \rightarrow 1D, D \rightarrow 1E, E \rightarrow 0E, E \rightarrow 1E, E \rightarrow 0F, F \rightarrow 0G, G \rightarrow \lambda\}.
				\end{align*}
				Wir begründen den Entwurf wieder über den äquivalenten NEA:
				
				\begin{center}
				\begin{tikzpicture}[->,>=stealth',shorten >=1pt,auto,node distance=2.8cm,
				semithick,every node/.style={scale=0.75}]
				\tikzstyle{every state}=[circle,text=black]
				
				\node[initial,state]			(S)             		{$S$};
				\node[state]					(A) [right of=S]		{$A$};
				\node[state]					(B) [right of=A]		{$B$};
				\node[state]					(C) [right of=B]		{$C$};
				\node[state]					(D) [right of=C]		{$D$};
				\node[state]					(E) [right of=D]		{$E$};
				\node[state]					(F) [right of=E]		{$F$};
				\node[state,accepting]			(G) [right of=F]		{$G$};
				
				\path
				(S)		edge [loop above]					node {0,1}	(S)
				(S)		edge 								node {1}	(A)
				(A)		edge  								node {0,1}	(B)
				(B)		edge 								node {0,1}	(C)
				(C)		edge 								node {0,1}	(D)
				(D)		edge								node {1}	(E)
				(E)		edge [loop above]					node {0,1}	(E)
				(E)		edge								node {0}	(F)
				(F)		edge 								node {0}	(G)
				;
				\end{tikzpicture}
				\end{center}
				
				Es ist klar, dass der NEA und die Grammatik $G_2$ äquivalent sind und dass der NEA $L_2$ erzeugt. Also ist $L_2 = L(G_2)$.
				
				\item
				Eine allgemeine Grammatik für die Sprache $L_3$ ist $G_3 = (\{S, A, X, Y, Z\}, \{0,1,2\}, P_3, S)$ mit
				\begin{align*}
					P_3 = &\{S \rightarrow XYZ, Z \rightarrow A2Z, 2A \rightarrow A2, YA \rightarrow A1Y, 1A \rightarrow A1, XA \rightarrow 0X, \\ & Z \rightarrow \lambda, Y \rightarrow \lambda, X \rightarrow \lambda\}.
				\end{align*}
				Die Idee ist, dass, wenn vor dem $Z$ eine 2 eingefügt wird, immer auch eine 1 vor dem $Y$ und eine 0 vor dem $X$ eingefügt wird. Dafür sorgt ein temporäres $A$, dass von hinten nach vorn durchgeschoben wird. Die letzten drei Regeln in $P_3$ sind dann dafür da, $X$, $Y$ und $Z$ aus dem Wort zu entfernen und das Wort $0^01^02^0 = \lambda$ zu ermöglichen.
				
				Die Ableitung des Wortes 000111222 ist damit:
				\begin{align*}
				S & \Rightarrow_{G_3} XYZ \Rightarrow_{G_3} XYA2Z \Rightarrow_{G_3} XA1Y2Z \\ & \Rightarrow_{G_3} 0X1Y2Z \Rightarrow_{G_3} 0X1Y2A2Z \Rightarrow_{G_3} 0X1YA22Z \Rightarrow_{G_3} 0X1A1Y22Z \Rightarrow_{G_3} 0XA11Y22Z \\ & \Rightarrow_{G_3} 00X11Y22Z \Rightarrow_{G_3} 00X11Y22A2Z \Rightarrow_{G_3}^2 00X11YA222Z \Rightarrow_{G_3} 00X11A1Y222Z \Rightarrow_{G_3}^2 00XA111Y222Z \\ & \Rightarrow_{G_3} 000X111Y222Z \Rightarrow_{G_3} 000111222
				\end{align*}
			\end{enumerate}
		\end{homeworkProblem}
	\end{document}
